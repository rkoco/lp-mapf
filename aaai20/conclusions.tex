\section{Conclusions and Future Work}
In this paper we proposed the first compilation of MAPF under the sum-of-costs assumption to ASP. We proposed a number of variants of the approach. In its first, basic form, the encoding is quadratic on the number of agents, just like existing approaches to ASP (e.g. \nbcite{ErdemKOS13}; \nbcite{GebserKKS14}), and like the encoding of MDD-SAT \cite{SurynekFSB16}, a state of the art SAT-based solver. We also propose an encoding that is linear on the number of agents, and show how we can benefit by running Dijkstra's algorithm during preprocessing time to generate a more compact encoding.

In our empirical evaluation on square grids, we observed that the linear encoding substantially outperforms the quadratic encoding. In general our approach outperforms the search-based and SAT-based state-of-the-art solvers we compared with as the number of agents increases. This suggests that ASP is a competitive approach for solving highly congested MAPF instances.

An important drawback of our approach, which only becomes apparent as larger grids are used, is grounding time. A potential (yet not elegant) line of improvement is to perform grounding manually.

Another potential line of future work is the integration of our approach to other algorithms like Windowed Hierarchical Cooperative A* \cite{Silver05}, since it is more efficient to generate shorter (though suboptimal) solutions on larger maps.
