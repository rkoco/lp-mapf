\section{Sum-of-Costs in ASP}
The encoding we presented in the previous section still does not produce cost-optimal solutions. Indeed, once fed into an ASP will return a model only if a solution with makespan $\mathtt{T}$ exists. In this section we explain how we can obtain solutions for sum-of-costs MAPF.

There is a natural way to encode sum-of-costs minimization: to minimize the number of actions performed by each agent before stopping at the goal. We noticed, however, that this yields an encoding whose size grows linearly with the size of the grid, $|V|$. This motivated us to look for a more compact encoding which would not depend on $|V|$. Even though, as we see in our empirical evaluation below, the grid-independent encoding performs better in practice, we describe both approaches here since the grid-dependent encoding is more natural and is a contribution on its own since sum-of-costs had not been encoded in ASP before.

\subsection{Grid-Dependent Encoding}
This encoding is similar to the approach used in MDD-SAT~\cite{SurynekFSB16}: the idea to minimize the actions performed by the agent at each cell before stopping at the goal. At a first glance one might think that we just need to count every action performed away from the goal and minimize this number. This approach, however does not work because a \Wait at the goal at time $t$ \emph{should} be counted if the agent will move away from the goal at some instant $t'$ greater than $t$.

To identify time instants at which we know the agent will not move away from the goal, we introduce the predicate $at\_goal\_back(a,t)$, which specifies that agent $a$ has reached the goal at time $t$  and will not move away in the future:
\begin{eqnarray*}\small
    at\_goal\_back(A,\mathtt{T}) &\leftarrow& agent(A), \\
    at\_goal\_back(A,T-1) &\leftarrow &at\_goal\_back(A,T),\\
    &&exec(A,wait,T-1).
  \end{eqnarray*}

Now we define predicate $cost$, such that thre is an atom of the form $cost(a,t,1)$ in the model whenever agent $a$ performes an action at time $t$ before stopping at the goal. First we express that moving an agent from a cell that is not the goal is penalized by one unit:
\begin{equation*}\small
cost(A,T,1) \leftarrow at(A,X,Y,T), not \; goal(A,X,Y).
\end{equation*}
Second, moving an agent away from the goal is also penalized by one:
\begin{equation*}\small
  \begin{split}
    cost(A,T,1) \leftarrow &at(A,X,Y,t), goal(A,X,Y), \\
    &exec(A,M,t), M \neq \Wait.
  \end{split}
  \end{equation*}

Third, if an agent performs a \Wait at the goal, but moves at a later time instant, then this is also penalized:
\begin{equation*}\small
  \begin{split}
    cost(A,T,1) \leftarrow &at(A,X,Y,t), goal(A,X,Y), \\
    &exec(A,wait,T), not \; at\_goal\_back(A,T).
  \end{split}
  \end{equation*}


Finally, via an optimization statement, we minimize the number of atoms of the form $cost(A,T,1)$ in the model:
\[ \#minimize \{C,T,A : cost(A,T,C)\}.\]

After grounding the number of rules is in $\Theta(|\mathcal{A}|\cdot|V| \cdot \mathtt{T})$.

\subsection{Grid-Independent Encoding}
For this encoding, we define the atom $optimal(a, c_a)$, for each agent $a\in \mathcal{A}$, where $c_a$ corresponds to the cost of the optimal path from $init(a)$ to $goal(a)$ ignoring both vertex and swap conflicts. Essentially $c_a$ is the result of solving a relaxation of the problem that ignores other agents. We compute such a value using Dijkstra's algorithm, before generating the encoding.

In contrast to the first encoding, we maximize the slack between the makespan $T$ and the time instant at which an agent has stopped at the goal, by simply adding:
\begin{equation*}\small
  \begin{split}
    penalty(A,T,1) \leftarrow &optimal(A,C), T > C, \\
    &at\_goal\_back(A,T-1).
\end{split}
\end{equation*}

Note that since no reference to the grid cells is made, the grounding generates a number of rules in $\Theta(|\mathcal{A}|\cdot\mathtt{T})$. Finally, we use the following maximization statement.
\[ \small \#maximize \{P,T,A : penalty(A,T,P)\}.\]

%Finally, the cost of the solution is $C = \mathtt{T} \times A - penalty $
