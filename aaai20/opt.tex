\section{Optimization}
To find a cost-optimal solution, we need to encode the minimization of the number of actions executed by the agents. We tried with two different encodings.

\subsection{Grid-dependent encoding}
This encoding is similar to the approach used in MDD-SAT: the idea to minimize the actions performed by the agent at each cell. Specifically there is an atom $cost(a,t,1)$ which specifies that agent $a$ has performed an action at time $t$. To define this atom, we need to specify three rule schemas:
\begin{enumerate}
    \item Moving an agent from a cell that is not the goal is penalized by one unit:
    \begin{align*} 
    cost(A,T,1) \leftarrow &at(A,X,Y,T), \\ 
    &not \; goal(A,X,Y).
    \end{align*}
    
    
    \item Moving an agent away from the goal is also penalized by one:
    \begin{align*} 
    cost(A,T,1) \leftarrow &at(A,X,Y,t), goal(A,X,Y), \\ 
    &exec(A,M,t), m \neq wait.
    \end{align*}

    
    \item If an agent performs a wait at the goal, but moves at a later \emph{time instant}, then is also penalized:
    \begin{align*} 
    cost(A,T,1) \leftarrow &at(A,X,Y,t), goal(A,X,Y), \\ 
    &exec(A,wait,T), not \; at\_goal\_back(A,T).
    \end{align*}

    
    Where $at\_goal\_back(a,t)$ specify that the agent $a$ at time $t$ has reached the goal and will not move away in the future. This is defined by the following rules:
    \begin{align*} 
    at\_goal\_back(A,\mathtt{T}) \leftarrow &agent(A). \\ 
    at\_goal\_back(A,T-1) \leftarrow &at\_goal\_back(A,T),\\ 
    &exec(A,wait,T-1).
    \end{align*}

    
    
    
\end{enumerate}
$\Theta(|V| \times \mathtt{T})$

The resulting number of rules and constraints grows linearly with the size of the grid and the number of agents.


Finally, via an optimization statement, we minimize the number of atoms of the form $cost(A,T,1)$ in the model:
\[ \#minimize \{C,T,A : cost(A,T,C)\}.\]


\subsection{Grid-independent encoding}
For this encoding, we define the atom $optimal(A, C)$, which specifies the cost of the optimal path of the agent $A$ ignoring the conflict constraint of the problem. This can be easily pre-computed via the Dijkstra algorithm for each agent separately.



In contrast to the first encoding, we maximize the slack between the makespan $T$ and the time instant at which an agent has stopped at the goal.We define the atom.



\begin{align*} 
    penalty(A,T,1) \leftarrow &optimal(A,C), T > C \\ 
    &at\_goal\_back(A,T-1).
\end{align*}

$\Theta(A \times \mathtt{T})$ The rule defining these atom doesn't make reference to the cells in the grid, resulting in an encoding that does not depend on the grid, and thus is linear in the number of agents. We add the optimization statement:

\[ \#maximize \{P,T,A : penalty(A,T,P)\}.\]

Finally, the cost of the solution is $C = \mathtt{T} \times A - penalty $




