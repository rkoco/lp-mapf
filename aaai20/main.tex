\def\year{2019}\relax
%File: formatting-instruction.tex
\documentclass[letterpaper]{article} % DO NOT CHANGE THIS
\usepackage{aaai19}  % DO NOT CHANGE THIS
\usepackage{times}  % DO NOT CHANGE THIS
\usepackage{helvet} % DO NOT CHANGE THIS
\usepackage{courier}  % DO NOT CHANGE THIS
\usepackage[hyphens]{url}  % DO NOT CHANGE THIS
\usepackage{graphicx} % DO NOT CHANGE THIS
\urlstyle{rm} % DO NOT CHANGE THIS
\def\UrlFont{\rm}  % DO NOT CHANGE THIS
\usepackage{graphicx}  % DO NOT CHANGE THIS
\frenchspacing  % DO NOT CHANGE THIS
\setlength{\pdfpagewidth}{8.5in}  % DO NOT CHANGE THIS
\setlength{\pdfpageheight}{11in}  % DO NOT CHANGE THIS


%PDF Info Is REQUIRED.
% For /Author, add all authors within the parentheses, separated by commas. No accents or commands.
% For /Title, add Title in Mixed Case. No accents or commands. Retain the parentheses.
 \pdfinfo{
/Title ()
/Author ()
} %Leave this
% /Title ()
% Put your actual complete title (no codes, scripts, shortcuts, or LaTeX commands) within the parentheses in mixed case
% Leave the space between \Title and the beginning parenthesis alone
% /Author ()
% Put your actual complete list of authors (no codes, scripts, shortcuts, or LaTeX commands) within the parentheses in mixed case.
% Each author should be only by a comma. If the name contains accents, remove them. If there are any LaTeX commands,
% remove them.

% DISALLOWED PACKAGES
% \usepackage{authblk} -- This package is specifically forbidden
% \usepackage{balance} -- This package is specifically forbidden
% \usepackage{caption} -- This package is specifically forbidden
% \usepackage{color (if used in text)
% \usepackage{CJK} -- This package is specifically forbidden
% \usepackage{float} -- This package is specifically forbidden
% \usepackage{flushend} -- This package is specifically forbidden
% \usepackage{fontenc} -- This package is specifically forbidden
% \usepackage{fullpage} -- This package is specifically forbidden
% \usepackage{geometry} -- This package is specifically forbidden
% \usepackage{grffile} -- This package is specifically forbidden
% \usepackage{hyperref} -- This package is specifically forbidden
% \usepackage{navigator} -- This package is specifically forbidden
% (or any other package that embeds links such as navigator or hyperref)
% \indentfirst} -- This package is specifically forbidden
% \layout} -- This package is specifically forbidden
% \multicol} -- This package is specifically forbidden
% \nameref} -- This package is specifically forbidden
% \natbib} -- This package is specifically forbidden -- use the following workaround:
% \usepackage{savetrees} -- This package is specifically forbidden
% \usepackage{setspace} -- This package is specifically forbidden
% \usepackage{stfloats} -- This package is specifically forbidden
% \usepackage{tabu} -- This package is specifically forbidden
% \usepackage{titlesec} -- This package is specifically forbidden
% \usepackage{tocbibind} -- This package is specifically forbidden
% \usepackage{ulem} -- This package is specifically forbidden
% \usepackage{wrapfig} -- This package is specifically forbidden
% DISALLOWED COMMANDS
% \nocopyright -- Your paper will not be published if you use this command
% \addtolength -- This command may not be used
% \balance -- This command may not be used
% \baselinestretch -- Your paper will not be published if you use this command
% \clearpage -- No page breaks of any kind may be used for the final version of your paper
% \columnsep -- This command may not be used
% \newpage -- No page breaks of any kind may be used for the final version of your paper
% \pagebreak -- No page breaks of any kind may be used for the final version of your paperr
% \pagestyle -- This command may not be used
% \tiny -- This is not an acceptable font size.
% \vspace{- -- No negative value may be used in proximity of a caption, figure, table, section, subsection, subsubsection, or reference
% \vskip{- -- No negative value may be used to alter spacing above or below a caption, figure, table, section, subsection, subsubsection, or reference

%% Our packages
\usepackage[algoruled,vlined,linesnumbered]{algorithm2e}
\usepackage{amsmath}
\usepackage{amssymb}
\usepackage{latexsym}
\usepackage[usenames,dvipsnames,svgnames,table]{xcolor}
\usepackage{units}
\usepackage{booktabs}

%% Our macros
\newcommand{\jb}[1]{{\small\sf \color{blue} Jorge: #1}}
\newcommand{\ch}[1]{{\small\sf \color{red} Carlos: #1}}
\newcommand{\rg}[1]{{\small\sf \color{orange} Rodrigo: #1}}

\newtheorem{definition}{Definition}
\newtheorem{theorem}{Theorem}
\newtheorem{lemma}{Lemma}
\newtheorem{corollary}{Corollary}

\newcommand{\citea}[1]{\citeauthor{#1}~(\citeyear{#1})}

%%%%%%%%%%%%%%%%%%%%%%%%%%%%%%%%%%%%%%%%%%%%%%%%%%%

%\setcounter{secnumdepth}{2}

% The file aaai19.sty is the style file for AAAI Press
% proceedings, working notes, and technical reports.
%
%\setlength\titlebox{2.5in} % If your paper contains an overfull \vbox too high warning at the beginning of the document, use this
% command to correct it. You may not alter the value below 2.5 in
\title{Solving Sum-of-Costs Multi-Agent Pathfinding with Answer-Set Programming}
%Your title must be in mixed case, not sentence case.
% That means all verbs (including short verbs like be, is, using,and go),
% nouns, adverbs, adjectives should be capitalized, including both words in hyphenated terms, while
% articles, conjunctions, and prepositions are lower case unless they
% directly follow a colon or long dash
\author{Paper \# 9802}
%\author{Written by AAAI Press Staff\textsuperscript{\rm 1}\thanks{Primarily Mike Hamilton of the Live Oak Press, LLC, with help from the AAAI Publications Committee}\\ \Large \textbf{AAAI Style Contributions by
%Pater Patel Schneider,} \\ \Large \textbf{Sunil Issar, J. Scott Penberthy, George Ferguson, Hans Guesgen}\\ % All authors must be in the same font size and format. Use \Large and \textbf to achieve this result when breaking a line
%\textsuperscript{\rm 1}Association for the Advancement of Artificial Intelligence\\ %If you have multiple authors and multiple affiliations
% use superscripts in text and roman font to identify them. For example, Sunil Issar,\textsuperscript{\rm 2} J. Scott Penberthy\textsuperscript{\rm 3} George Ferguson,\textsuperscript{\rm 4} Hans Guesgen\textsuperscript{\rm 5}. Note that the comma should be placed BEFORE the superscript for optimum readability
%2275 East Bayshore Road, Suite 160\\
%Palo Alto, California 94303\\
%publications19@aaai.org % email address must be in roman text type, not monospace or sans serif
%}

% \newcommand{\C}[1]{\ensuremath{\mathcal{#1}}\xspace}
% \newcommand{\R}{\C{R}}
% \newcommand{\D}{\C{D}}
% %\newcommand{\nnfev}{\models_{\mathsf{rg}}}
% \newcommand{\nnfev}{\models{\kern-.8em\lower.9ex\hbox{\rm\scriptsize\textsf{rg}}}}
% %u\kern-.56em\lower.3ex\mbox{2}

% \newcommand{\Deff}{\ensuremath{\C{D}_\mathit{eff}}\xspace}
% \newcommand{\Dpr}{\ensuremath{\C{D}_\mathit{ap}}\xspace}
% \newcommand{\Dnpr}{\ensuremath{\C{D}_\mathit{nap}}\xspace}
% \newcommand{\Duna}{\ensuremath{\C{D}_\mathit{una}}\xspace}
% \newcommand{\Dso}{\ensuremath{\C{D}_{S_0}}\xspace}
% \newcommand{\Kinit}{\ensuremath{\C{K}_\mathit{init}}\xspace}

% % first-order aliases
% \newcommand{\x}{\ensuremath{\vec{x}}}
% \newcommand{\y}{\ensuremath{\vec{y}}}
% \newcommand{\Ex}[1]{\ensuremath{(\exists #1)\,}}
% \newcommand{\fall}[1]{\ensuremath{(\forall #1)\,}}
% \newcommand{\Exd}[1]{\ensuremath{(\exists #1).\,}}
% \newcommand{\falld}[1]{\ensuremath{(\forall #1).\,}}
% \newcommand{\Lang}{\C{L}\xspace}
% \newcommand{\LangFO}{\C{L}_{FO}\xspace}
% \newcommand{\Vars}{\ensuremath{\mathrm{Vars}}}
% \newcommand{\QPrefix}{\ensuremath{\mathsf{QPrefix}}\xspace}
% \newcommand{\Tempmembers}{\ensuremath{\Upsilon}\xspace}
% \newcommand{\ground}{\ensuremath{\mathsf{ground}}\xspace}

% \newcommand{\SCond}{\ensuremath{\mathsf{SensedCond}}\xspace}

% \newcommand{\subst}{\ensuremath{\mathtt{subst}}}

% \newcommand{\eqdef}{\ensuremath{\overset{\text{def}}{=}}}

\newcommand{\proofstring}{Proof}
%\newcommand{\proof}{\noindent\textbf{\proofstring:}\xspace}
\newcommand{\proofsketchstring}{Proof sketch}
\newcommand{\proofsketch}{\noindent\textbf{\proofsketchstring:}\xspace}
\newenvironment{pf}[1][]{\noindent\textbf{\proofstring\ifthenelse{\equal{#1}{}}{:}{~(#1) :}}\xspace}{\hfill\QED\medskip\par}
\newenvironment{pfsketch}[1][]{\noindent\textbf{\proofsketchstring\ifthenelse{\equal{#1}{}}{:}{~(#1) :}}\xspace}{\hfill\QED\medskip\par}

% \newcommand{\base}{\noindent\textit{Base case:}\xspace}
% \newcommand{\bases}{\noindent\textit{Base cases:}\xspace}
% \newcommand{\induction}{\noindent\textit{Induction:}\xspace}
\newcommand{\QED}{\hfill$\blacksquare$\vspace{0pt}}
% \newcommand{\GIF}[3]{\ensuremath{\mathbf{if}~#1~\mathbf{then}~#2~\mathbf{else}~#3~\mathbf{endif}}}
% \newcommand{\GWHILE}[3][]{\ensuremath{\mathbf{while}_{#1}~#2~\mathbf{do}~#3~\mathbf{endwhile}}}
% \newcommand{\scomment}[1]{{\bfseries [\textsl{#1}]}}
% \newcommand{\golog}{Golog\xspace}
% \newcommand{\eagle}{\textsc{EaGle}\xspace}
% \newcommand{\predeq}{\sqsubseteq}
% \newcommand{\pred}{\sqsubset}



% \newcommand{\SC}{Situation Calculus\xspace}


\newcommand{\angled}[1]{\ensuremath{\langle #1\rangle}}
%\newcommand{\comment}[1]{\textbf{[#1]}}

%% LTL operators
\newcommand{\final}{\ensuremath{\mathsf{final}}\xspace}
\newcommand{\true}{\ensuremath{\mathsf{true}}\xspace}
\newcommand{\false}{\ensuremath{\mathsf{false}}\xspace}
\newcommand{\until}{\ensuremath{\operatorname{\mathsf{U}}}}
\newcommand{\release}{\ensuremath{\operatorname{\mathsf{R}}}}
%\newcommand{\next}{\ensuremath{{\scriptstyle \bigcirc}}}

%\def\next{\raise1.4pt\hbox{$\scriptstyle\bigcirc$}}

\newcommand{\Al}{\ensuremath{\Box}}
\newcommand{\Ev}{\ensuremath{\Diamond}}

\newcommand{\acite}[1]{\citeauthor{#1}~\shortcite{#1}}
\newcommand{\nbcite}[1]{\citeauthor{#1}~\citeyear{#1}}
%\newcommand{\shortcite}[1]{\cite{#1}}

% theapa alias:
%\newcommand{\acite}[1]{\citeauthor{#1}~\cite{#1}}
%\newcommand{\acite}[1]{\cite{#1}}
%\newcommand{\citeauthor}[1]{\cite{#1}}

%% natbib aliases:
%\newcommand{\acite}[1]{\citet{#1}}
\makeatletter
%\DeclareRobustCommand{\cite}{\@ifnextchar<{\@jbcite}{\@jbcite<>}}
%\def\@jbcite<#1>{\@ifnextchar[{\@@jbcite<#1>}{\@@jbcite<#1>[]}}
%\def\@@jbcite<#1>[#2]#3{\protect\citep[#1][#2]{#3}}
%\def\citeS{\@ifnextchar[{\@jbciteS}{\@jbciteS[]}}
%\def\citeR#1{\citealp{#1}}
%\def\@jbciteS[#1]#2{\citeauthor{#2}'s #1 \cite{#2}}
%\def\@jbciteS[#1]#2{arg1=#1 arg2=#2}}
\makeatother

%\setcitestyle{round}


\newcommand{\Paragraph}[1]{\noindent\textbf{#1}\hspace{.8em}}

\newcommand{\PBS}[1]{\let\temp=\\#1\let\\=\temp}
\newcommand{\fs}[1]{\fontsize{#1}{#1}\selectfont}
\newenvironment{program}{\smallskip\begin{singlespaced}\begin{small}\begin{alltt}}%
  {\end{alltt}\end{small}\end{singlespaced}\smallskip}



\newcommand{\bestMetric}{\ensuremath{\mathit{bestMetric}}}


\newcommand{\strips}{\textsc{STRIPS}\xspace}
\newcommand{\adl}{\textsc{ADL}\xspace}

% complexity classes

\newcommand{\complexityclass}[1]{{\rm\textsf{#1}}\xspace}
\newcommand{\Pclass}{\complexityclass{P}}
\newcommand{\PSPACE}{\complexityclass{PSPACE}}
\newcommand{\NPSPACE}{\complexityclass{NPSPACE}}
\newcommand{\APX}{\complexityclass{APX}}
\newcommand{\EXPSPACE}{\complexityclass{EXPSPACE}}
\newcommand{\EXPTIME}{\complexityclass{EXPTIME}}
\newcommand{\NP}{\complexityclass{NP}}


\newcommand{\Succ}{\ensuremath{Succ}}

% Other aliases for algorithms
\newcommand{\Up}{\ensuremath{\mathit{up}}\xspace}
\newcommand{\Down}{\ensuremath{\mathit{down}}\xspace}
\newcommand{\Left}{\ensuremath{\mathit{left}}\xspace}
\newcommand{\Right}{\ensuremath{\mathit{right}}\xspace}
\newcommand{\Wait}{\ensuremath{\mathit{wait}}\xspace}



\newcommand{\techreport}[1]{#1}
\newcommand{\shortpaper}[1]{}

\newcommand{\smallboldfacenumber}{\tiny\bfseries}
\newcommand{\update}{\texttt{Update()}\xspace}
\newcommand{\extractbest}{\texttt{Extract-Best()}\xspace}
\newcommand{\snext}{\ensuremath{s_{next}}}
\newcommand{\sstart}{\ensuremath{s_{\mathit{start}}}\xspace}
\newcommand{\search}{\ensuremath{\mathit{search}}\xspace}
\newcommand{\Next}{\ensuremath{\mathit{next}}\xspace}
\newcommand{\saux}{\ensuremath{s_{\mathit{aux}}}\xspace}
\newcommand{\sinit}{\ensuremath{s_{\mathit{init}}}\xspace}
\newcommand{\sgoal}{\ensuremath{s_{\mathit{goal}}}\xspace}
\newcommand{\scurrent}{\ensuremath{s_{\mathit{current}}}\xspace}
\newcommand{\parent}{\ensuremath{\mathit{parent}}}
\newcommand{\counter}{\ensuremath{\mathit{counter}}}
\newcommand{\restart}{\ensuremath{\mathit{restart}}}
\newcommand{\updated}{\ensuremath{\mathit{updated}}\xspace}
\newcommand{\decrease}{\ensuremath{\mathit{decrease}}\xspace}
\newcommand{\pathtogoal}{\ensuremath{\mathit{pathToGoal}}\xspace}
\newcommand{\Tinc}{\ensuremath{E^{\mathit{inc}}}\xspace}
\newcommand{\Tdec}{\ensuremath{E^{\mathit{dec}}}\xspace}
\newcommand{\Tupd}{\ensuremath{V^{\mathit{upd}}}\xspace}




 \begin{document}

\maketitle

\begin{abstract}
Solving a Multi-Agent Pathfinding (MAPF) problem involves finding non-conflicting paths that lead a number of agents to their goal location. In the sum-of-costs variant of MAPF, one is also required to minimize the total number of moves performed by agents before stopping at the goal. Not suprisingly, since MAPF is combinatorial, a  number of compilations to Satisfiability solving (SAT) and Answer Set Programming (ASP) exist. In this paper, we propose the first family of compilations to ASP that solve sum-of-costs MAPF over 4-connected grids. Unlike existing compilations we are aware of, including to SAT, our encoding is the first that, after grounding, produces a number of clauses that is \emph{linear} on the number of agents. In addition, the representation of the optimization objective is also carefully written, such that its size after grounding does not depend on the size of the grid. In our experimental evaluation, we show that our approach outperforms search- and SAT-based sum-of-costs sMAPF solvers when grids are congested with agents.
%Like makespan-optimal approaches, our algorithm searches for cost-optimal solutions with increasing makespan. When a solution is found a provably correct upper bound on the maximum makespan at which a true cost-optimal solution exists is computed, and the solver is rerun once more.
\end{abstract}



\section{Introduction}

\section{Background}

\subsection{Multi-Agent Pathfinding over 4-Connected Grids}
We use the standard definition of MAPF for 4-connected grids. An $n\times m$ grid is defined by a pair $(G,O)$, where $G=\{(i,j):i\in\{1,\ldots,n\}, j\in\{1,\ldots,m\}\}$ are the cells of the grids and $O\subseteq G$ is a set of obstacle cells. In addition, we have $k$ agents, each of which is associated with an initial cell $I_k$ and a goal cell $G_k$. Intuitively each agent, at any time can perform one of five actions \{\Up,\Down,\Left,\Right,\Wait\}, the first four of which move the agent one cell in the desired direction. \Wait is an action that does not move the agent. When an agent executes a sequence of actions it generates a \emph{path}. A path is a sequence of nodes $u_1u_2\ldots u_m$, where $u_{i+1}$ is a neighbor of $u_i$ or $u_{i+1}=u_i$.  Given a path $u_0u_1\ldots u_m$, we say that the cell visited at time step $t$ is $u_t$ if $t\leq n$ or $u_m$ if $t>m$. Two paths are \emph{conflict-free} if (1) they are such that they do not visit the same cell at the same time step, and (2) agents never `swap' each other; that is they do not exchange (neighboring) cells in consecutive time steps.  A \emph{solution} to a MAPF problem is a list of $k$ pairwise \emph{conflict-free paths}, such that the $i$-th path (for agent $i$) starts in $I_i$ and ends in $G_i$. The cost of a path of size $n$ is $n-1$ (since we assume that actions have a cost of 1). The cost of a solution is the sum of the costs of the paths in the solution. The \emph{makespan} of a solution is the length of the longest path in the solution minus one. Two optimality criteria are usually considered for MAPF. A solution is cost-optimal (also referred to as \emph{optimal under sum-of-costs}) when no other solution exists with a lower cost. A solution is makespan-optimal when no other solution exists with a lower makespan.

\subsection{Answer-Set Programming}
ASP \cite{paper-de-lifschitz-what-is-ASP} is a logic based framework for solving optimization problems. For space limitations, here we describe a subset of an ASP standard that is relevant to this paper. 

An ASP \emph{basic program} is a set of rules of the form:
\begin{equation}\label{asprule}
p\leftarrow q_1,q_2,\ldots,q_n.
\end{equation}
where $n\geq 0$, and $p$, $q_1,\ldots,q_n$ are so-called \emph{atoms}. The intuitive interpretation of this rule is as follows ``p is true/provable if so are $q_1,q_2,\ldots,q_n$''. When $n=0$ the rule \eqref{asprule} is called a \emph{fact}.

A \emph{model} of a basic ASP program is a set of atoms $M$ that intuitively contains all and only the atoms that are provable. An important syntactic element relevant to our paper is the so-called \emph{negation as failure}. Rules containing such negated atoms look like:
\begin{equation}\label{asprule}
p\leftarrow q_1,q_2,\ldots,q_n, not\: r_1,not\: r_2, \ldots, not\: r_k .
\end{equation}
The set of atoms $M$ is a model of a program $P$ containing such rules if the \emph{reduct} of $\Pi$ with respect to $M$, denoted as $\Pi^M$ also has $M$ as a model. Intuitively $P^M$ is a basic program which results from simplifying away every occurrence of a $not\:p$ in the obvious way (i.e., eliminating the rule if $p$ is in the model $M$, and removing $not\:p$ from the rule if $p$ is not in the model $M$). For example $M=\{p\}$ is a model of $\Pi=\{p\leftarrow not\: q\}$, since $\Pi^M=\{p\leftarrow \}$ has $M$ as a model. $M'=\{q\}$ is not a model of $\Pi$ since $\Pi^{M'}=\{\}$ does not have $M'$ as a model.

Another relevant type of rule is
\begin{equation}
    |\{p_1,p_2,\ldots,p_n\}|=1 \leftarrow q_1,q_2,\ldots,q_n.
\end{equation}
which states that only one among $\{p_1,\ldots,p_m\}$ is in a model $M$ if so are $\{q_1,\ldots,q_n\}$. In addition, the constraint:
\begin{equation}
    \leftarrow p_1,p_2,\ldots,p_n 
\end{equation}
(best pronounced as not all of $p_1,\ldots,p_n$) states that a model cannot contain the set $\{p_1,p_2,\ldots,p_n\}$.

A very relevant feature of ASP is that programs can also contain a minimization statement expressing that one is looking for a model that minimizes a certain criteria. We introduce these constructs later when we see the details of our compilation. An ASP solver receives a program as input and outputs a model.

Finally, ASP programs usually use variables to represent schemas of rules. As such $p(X)\leftarrow q(X)$ represents a family of rules where $X$ moves over all the elements constructible with the syntactic elements mentioned in the program (the so-called Herbrand base). We omit further details for lack of space. What is very relevant however, is that an ASP solver will \emph{ground} a program containing variables, effectively generating all instances of these rule schemas before finding a model.


experimentos:

grillas (la campana)
\section{A Basic Translation of MAPF to ASP}
We are now ready to describe our compilation of sum-of-costs MAPF to ASP. As we have mentioned above, this is the first compilation to ASP that handles sum-of-costs. Besides that aspect of novelty, the basic compilation that we present here is similar in many aspects to \citeauthor{ErdemKOS13}'s compilation \shortcite{ErdemKOS13} to ASP, and, in some aspects simlar to the MAPF-to-SAT compilation of \acite{Surynek14}. Below we are specific about these similarities.

As most compilations of planning problems into SAT/ASP, the makespan of the compilation is a parameter, which below we call $\mathtt{T}$.
%\begin{itemize}
%    \item $a$ refers to an agent.
%    \item $x,y$ refers to a position in the grid.
%    \item $m$ indicates a move or action. Where $m \in \{up,down,left,right,wait\}$.
%\end{itemize}

\subsubsection{Atoms}
We use the following atoms:
\begin{itemize}
\item $agent(a)$: to express that $A$ is an agent,
\item $goal(a,x,y)$: specifies that the goal cell for agent $a$ is $(x,y)$,
\item $obstacle(x,y)$: specifies that cell $(x,y)$ is an obstacle,
\item $at(a,x,y,t)$: specifies that agent $a$ is at $(x,y)$ at time $t$,
\item $exec(a,m,t)$: specifies that agent $a$ executes move $m$ at time $t$,
\item $at\_goal(a,t)$: specifies that agent $a$ is at the goal at time $t$,
\item $time(t)$: $t$ is a time instant,
\item $move(m)$: $m$ is a move.
\end{itemize}
Finally, we use atoms $rangeX(x)$ and $rangeY(y)$ to specify that $(X,Y)$ is within the limits of the grid.

\subsubsection{Instance Specification}
To specify a particular MAPF instance, we define facts for atoms of the form $agent(a)$, for each $a\in A$, $obstacle(x,y)$ for each $(x,y)$ that is marked as an obstacle in the grid, $rangeX(x)$ for each $x\in\{1,\ldots,w\}$, where $w$ is the width of the grid, and $rangeY(y)$ for each $y\in\{1,\ldots,h\}$, where $h$ is the height of the grid. Additionally, we define the initial cells for each agent, adding one fact of the form $at(a,x_a,y_a,0)$ for each agent $a\in \mathcal{A}$, where $(x_a,y_a)=init(a)$. Furthermore add an atom of the form $time(t)$ for every $t\in\{1,\dots,\mathtt{T}\}$. The number of rules needed to encode a MAPF instance is therefore in $\Theta(|\mathcal{A}| + \mathtt{T} + |V|)$.

\subsubsection{Effects}
To encode the effects of the five actions, we use a single rule written as follows:
\begin{equation}\small\label{encoding:effectsone}
\begin{split}
at(A,X,Y,T) \leftarrow &exec(A,M,T-1),\\&at(A,X',Y',T-1), \\&delta(M,X',Y',X,Y).
\end{split}
\end{equation}
which specifies that if agent $A$ is at position $(X',Y')$ in time instant $T-1$, then it will be in position $(X,Y)$ in time instant $T$ iff $(X,Y)$ and $(X',Y')$ satisfy predicate $delta$. Auxiliary predicate $delta$ is used to establish a relation between $(X,Y)$ and $(X',Y')$ given a certain move $M$ in the following way:
\begin{equation}\small\label{encoding:delta}
\begin{split}
&delta(\Right,X,Y,X+1,Y) \leftarrow rangeX(X), rangeY(Y),\\
&delta(\Left,X,Y,X-1,Y) \leftarrow rangeX(X), rangeY(Y),\\
&delta(\Up,X,Y,X,Y+1) \leftarrow rangeX(X), rangeY(Y),\\
&delta(\Down,X,Y,X,Y-1) \leftarrow rangeX(X), rangeY(Y),\\
&delta(\Wait,X,Y,X,Y) \leftarrow rangeX(X),rangeY(Y).
\end{split}
\end{equation}
A grounding time predicate $delta$ results in 5 rules per each position of the grid. This defines that the total number of grounded instances for rule \eqref{encoding:effectsone} is proportional to the size of the grid, the number of agents and the number of time instants. The total number of instances for rules of the form \eqref{encoding:effectsone} and \eqref{encoding:delta} is in $\Theta(|\mathcal{A}|\cdot |V| \cdot \mathtt{T})$.

\subsubsection{Parallel move execution}
We need to encode that each agent performs exactly one move at each time instant. To do this we write the following rule:
\begin{equation}\small\label{encoding:delta}
\begin{split}
|\{exec(A,M,T-1) : move(M) \}| = 1 \leftarrow &time(T),\\&agent(A).
\end{split}
\end{equation}
Upon grounding the number of instances of this rule is in $\Theta(|\mathcal{A}|\cdot \mathtt{T})$.
\subsubsection{Legal positions}
We need to express that the agents move through the vertices in the graph; that is, they cannot exit the grid or visit an obstacle cell. We do so using the following three rules:
\begin{equation}\label{encoding:legal}\small
    \begin{split}
&\leftarrow at(A,X,Y,T), not\: rangeX(X),\\
&\leftarrow at(A,X,Y,T), not\: rangeY(Y),\\
&\leftarrow at(A,X,Y,T), obstacle(X,Y). %no hay robots encima de obstaculos
    \end{split}
\end{equation}
The total number of grounded rules for the rules of form \eqref{encoding:legal} is in $\Theta(|\mathcal{A}| \cdot |V| \cdot \mathtt{T})$, since it depends on the number of atoms of the form $at$, $obstacle$, $rangeX$, and $rangeY$.

\subsubsection{Vertex Conflicts}
To express that no agents can be at the same vertex we use the following constraint, which is similar to those used in the encoding to ASP by \acite{ErdemKOS13} and \acite{SurynekFSB16} \textbf{revisar si es la cita correcta a Surynek}.
\begin{equation}\small\label{encoding:vertexone}
    \leftarrow at(a,x,y,t), at(a',x,y,t)
\end{equation}
The number of instances for this rule \eqref{encoding:vertexone} after grounding is $\Theta(|\mathcal{A}|^2 \cdot |V| \cdot \mathtt{T})$. Note that this is the first rule so far whose instantiation is quadratic on the number of agents. This motivates the improvement we present later in the following section.
\subsubsection{Swap Conflicts}
No pair of agents can swap their positions. We express this avoiding horizontal and vertical swaps using the following constraints.
\begin{equation}\small\label{encoding:swapone}
  \begin{split}
        \leftarrow &at(A,X+1,Y,T-1), at(A',X,Y,T-1),\\
        &at(A,X,Y,T), at(A',X+1,Y,T). \text{\% horizontal swap} \\
        \leftarrow &at(A,X,Y+1,T-1), at(A',X,Y,T-1),\\
        &at(A,X,Y,T), at(A',X,Y+1,T). \text{\% vertical swap}
  \end{split}
\end{equation}
The number of ground rules for \eqref{encoding:swapone} is in $\Theta(|\mathcal{A}|^2 \cdot |V| \cdot \mathtt{T})$.

\subsubsection{Goal Achievement}
We specify with a constraint that no agent is away from its goal at time $\mathtt{T}$:
\begin{equation}\small\label{encoding:goal}
  \begin{split}
&at\_goal(a,t) \leftarrow at(a,x,y,t), goal(a,x,y),\\
&\leftarrow agent(a), not \; at\_goal(a,\mathtt{T}).
\end{split}
\end{equation}
The number of instances for rules \eqref{encoding:goal} is $\Theta(\mathcal{A} \cdot |V| \cdot \mathtt{T})$.
\subsubsection{Size of Basic Encoding}
After grounding, it follows that the size of the total encoding is in $\Theta(|\mathcal{A}|^2 \cdot |V| \cdot \mathtt{T})$. That is, it is quadratic in the number of agents, linear in the size of the grid, and linear in the makespan parameter $\mathtt{T}$.

\section{A Linear Encoding}
The encoding we have proposed is quadratic in the number of agents. In this section we show how to make it linear by introducing new atoms to the encoding. Specifically, we introduce the following atoms:
\begin{itemize}
    \item $rt(x,y,t)$ (resp.\ $lt(x,y,t)$) which specifies that the edge between $(x,y)$ and $(x,y+1)$ was traversed by \emph{some} agent at time $t$ from left to right (resp.\ from right to left).
    \item $ut(x,y,t)$ (resp.\ $dt(x,y,t)$) which specifies that the edge between $(x,y)$ and $(x,y+1)$ was traversed upwards (resp.\ downwards) by some agent at time $t$.
    \item $st(x,y,t)$, indicates that some agent stayed at $(x,y)$, that is, it  performed a wait action at time $t$.
\end{itemize}

The dynamics of these atoms are defined using one rule with variables. The rule for $rt$ is:
\begin{equation}\small\label{encoding:rt}
  rt(X,Y,T) \leftarrow exec(A,right,T), at(A,X,Y,T),
\end{equation}
while the rule for $st$ is:
{\small\begin{equation*}
  st(X,Y,T) \leftarrow at(A,X,Y,T), exec(A,wait,T).
\end{equation*}}
We omit the rules for $dt$, $lt$, and $ut$ since they are analogous to \eqref{encoding:rt}.

Using these predicates we now express the fact that a single cell cannot be entered at the same time instant by two different agents. This requires six rules each of which corresponds to a pairs of actions in $\{\Right,\Left,\Up,\Down\}$. For example, the following rule expresses that $(X,Y)$ cannot be entered by an agent performing a \Down action at the same time that is entered by another agent performing  \Up:
{\small\begin{equation*}
  \leftarrow lt(X,Y,T), dt(X,Y,T)
\end{equation*}}
It is easy to verify that these rules do not mention pairs of different agents, unlike \eqref{encoding:vertexone} and \eqref{encoding:swapone}, and as such after grounding we end with $\Theta(|\mathcal{A}|\cdot|V|\cdot\mathtt{T})$ rules, and therefore the resulting encoding is linear in $|\mathcal{A}|$.

\section{Using Search to Reduce the Atoms}
We can exploit our run of Dijkstra's algorithm during preprocessing time to generate an even smaller encoding by replacing rule~\ref{encoding:effectsone} by the following rule.
\begin{equation}\small
  \begin{split}
    at(A,X,Y,T) \leftarrow &at(A,X',Y',t), exec(A,M,T), \cup B, \\
    &cost\_to\_go(A,X,Y,C), T + C <= \mathtt{T},
  \end{split}
\end{equation}
where $cost\_to\_go(A,X,Y,C)$ specifies that $C$ is the minimum number of actions needed to go from $(X,Y)$ to the goal of agent $A$. This way we can ignore the generation of rules to positions that will not reach the goal, generating a much more compact encoding. This idea is related to the use of MDDs graphs in MDD-SAT \cite{SurynekFSB16}, but does not require the generation of the MDD, so it is conceptually simpler.

\section{Sum-of-Costs in ASP}
The encoding we presented in the previous section still does not produce cost-optimal solutions. Indeed, once fed into an ASP will return a model only if a solution with makespan $\mathtt{T}$ exists. In this section we explain how we can obtain solutions for sum-of-costs MAPF.

There is a natural way to encode sum-of-costs minimization: to minimize the number of actions performed by each agent before stopping at the goal. We noticed, however, that this yields an encoding whose size grows linearly with the size of the grid, $|V|$. This motivated us to look for a more compact encoding which would not depend on $|V|$. Even though, as we see in our empirical evaluation below, the grid-independent encoding performs better in practice, we describe both approaches here since the grid-dependent encoding is more natural and is a contribution on its own since sum-of-costs had not been encoded in ASP before.

\subsection{Grid-Dependent Encoding}
This encoding is similar to the approach used in MDD-SAT~\cite{SurynekFSB16}: the idea to minimize the actions performed by the agent at each cell before stopping at the goal. At a first glance one might think that we just need to count every action performed away from the goal and minimize this number. This approach, however does not work because a \Wait at the goal at time $t$ \emph{should} be counted if the agent will move away from the goal at some instant $t'$ greater than $t$.

To identify time instants at which we know the agent will not move away from the goal, we introduce the predicate $at\_goal\_back(a,t)$, which specifies that agent $a$ has reached the goal at time $t$  and will not move away in the future:
{\small\begin{eqnarray*}
    at\_goal\_back(A,\mathtt{T}) &\leftarrow& agent(A), \\
    at\_goal\_back(A,T-1) &\leftarrow &at\_goal\_back(A,T),\\
    &&exec(A,wait,T-1).
  \end{eqnarray*}}

Now we define predicate $cost$, such that thre is an atom of the form $cost(a,t,1)$ in the model whenever agent $a$ performes an action at time $t$ before stopping at the goal. First we express that moving an agent from a cell that is not the goal is penalized by one unit:
\begin{equation*}\small
cost(A,T,1) \leftarrow at(A,X,Y,T), not \; goal(A,X,Y).
\end{equation*}
Second, moving an agent away from the goal is also penalized by one:
{\small \begin{equation*}\small
  \begin{split}
    cost(A,T,1) \leftarrow &at(A,X,Y,t), goal(A,X,Y), \\
    &exec(A,M,t), M \neq \Wait.
  \end{split}
  \end{equation*}

Third, if an agent performs a \Wait at the goal, but moves at a later time instant, then this is also penalized:
\begin{equation*}\small
  \begin{split}
    cost(A,T,1) \leftarrow &at(A,X,Y,t), goal(A,X,Y), \\
    &exec(A,wait,T), not \; at\_goal\_back(A,T).
  \end{split}
  \end{equation*}


Finally, via an optimization statement, we minimize the number of atoms of the form $cost(A,T,1)$ in the model:
\[ \#minimize \{C,T,A : cost(A,T,C)\}.\]

After grounding the number of rules is in $\Theta(|\mathcal{A}|\cdot|V| \cdot \mathtt{T})$.

\subsection{Grid-Independent Encoding}
For this encoding, we define the atom $optimal(a, c_a)$, for each agent $a\in \mathcal{A}$, where $c_a$ corresponds to the cost of the optimal path from $init(a)$ to $goal(a)$ ignoring both vertex and swap conflicts. In other words, $c_a$ is the result of solving a relaxation of the problem that ignores other agents. We compute such a value using Dijkstra's algorithm, before generating the encoding.

In contrast to the first encoding, we maximize the slack between the makespan $T$ and the time instant at which an agent has stopped at the goal, by simply adding:
\begin{equation*}\small
  \begin{split}
    penalty(A,T,1) \leftarrow &optimal(A,C), T > C, \\
    &at\_goal\_back(A,T-1).
\end{split}
\end{equation*}

Note that since no reference to the grid cells is made, the grounding generates a number of rules in $\Theta(|\mathcal{A}|\cdot\mathtt{T})$. Finally, we use the following maximization statement.
{\small\[ \small \#maximize \{P,T,A : penalty(A,T,P)\}.\]}

%Finally, the cost of the solution is $C = \mathtt{T} \times A - penalty $

%\section{Makespan as a Bound}
% To guarantee that our model will yield a solution that is \emph{sum-of-costs} optimal we need to provide a maximum $makespan$ $T$ big enough.

% In figure , we show an example where the increase of the \emph{makespan} returns better \emph{sum-of-costs} solutions. The problem has 3 agents:  $a_1$ needs to go from $(0,1)$ to $(3,1)$. The agents $a_2$ and $a_3$ are already at the goal on the positions $(1,1)$ and $(2,1)$ respectively. We compare the minimum \emph{sum-of-costs} solution with two different \emph{makespans}:

% \begin{itemize}
%     \item $\mu=3$. The optimal \emph{sum-of-costs} solution involves moving $a_2$ and $a_3$ out of their goal. The costs for each agent are: $cost(a_1) = 3$, $cost(a_2) = 2$ and $cost(a_3) = 3$, which gives us \emph{sum-of-costs} $=8$.
%     \item $\mu=5$. The optimal \emph{sum-of-costs} solution only needs to move $a_1$, dodging the locations occupied by the other agents: \emph{sum-of-costs} $=cost(a_1) = 5$
% \end{itemize}



% \emph{makespan} $m_{opt}$ that guarantees a cost-optimal solution given an initial solution.

% \begin{lemma}
% Given an initial solution with cost $\sigma$. We can bound $m_{opt}$ as:
% \[
%     m_{opt} \leq  \sigma - \min\limits_{i=\{1..k\} }{\sum_{\substack{j=1 \\ j \neq i}}^{k} c^*(a_j)}
% \]
% Where $c^*(a_j)$ is the optimal cost of the path $a_j$ ignoring the conflicts with other agents.
% \end{lemma}

\subsection{Finding Cost-Optimal Solutions}
The encoding proposed so far can find the minimum sum-of-cost solution for a given makespan. We still need to define how to find a true cost-optimal solution.

Following the approach used for SAT encodings for planning \cite{KautzS92}, in our approach we attempt to solve instances for increasing makespan $\mathtt{T}$, until a solution, say $sol_{min}$, is found. Two observations with this process are important. First, we do not need to start increasing $\mathtt{T}$ from 1. As mentioned above, at preprocessing time, for each agent we compute cost the cost $c^*_a$ which ignores other agents. The makespan of any solution must be at least $\max_{a\in\mathcal{A}} c^*_a$ so this can be the inferior limit of our iteration.

Second, let $sol_{min}$ be the solution that is found first. Unfortunately, $sol_{min}$ is a makespan-optimal solution but not necessarily a cost-optimal solution. Now we can compute a bound for the largest makespan $\mathtt{T}_{max}$ at which the cost-optimal solution is found, using the following theoretical result first proposed by \acite{SurynekFSB16}:
\begin{theorem}[\nbcite{SurynekFSB16}]\label{thm:optimal}
Let $sol_{min}$ be the makespan-optimal solution for MAPF problem $P$, let $sol^-$ denote a solution to $P$ that ignores all conflicts, and let $\mathtt{T}^-$ denote its makespan. Then the makespan of the cost-optimal solution is at most at $\mathtt{T}_{max}=\mathtt{T}^- + c(sol_{min})-c(sol^-)-1$.
\end{theorem}
Thus after we find the first solution $sol_{min}$, we run the solver again for makespan $\mathtt{T}_{max}$ given by Theorem~\ref{thm:optimal}. The approach described in this section was recently evaluated by \acite{BartakS19} for their Picat-based MAPF solver.


%First we calculate the shortest path for each agent ignoring the conflicts with other agents. The maximum cost and sum of costs of these paths provide a lower bound for the makespan $\mathtt{T}_{min}$ and \emph{sum-of-costs} respectively. Then we define and start solving an ASP program with one of the encodings defined before using $\mathtt{T}=\mathtt{T}_{min}$ as the makespan. If there's no solution we increase $\mathtt{T}$ by 1 and repeat the process.

%If the program returns a valid solution then as it was shown in Figure ~\ref{fig:makespancost} it's not guaranteed that is \emph{sum-of-cost} optimal.

%The using (bartak)

%So we run the program one last time using $\mathtt{T}=C(M) - 1 - C^-$ and return that solution.


% \begin{algorithm}
% \DontPrintSemicolon
% \KwIn{A MAPF problem}
% $\mathtt{T} \gets \max\limits_{i=\{1..K\} }{c^*(a_i)}$\;
% $C^- \gets \sum_{\substack{i=1}}^{k} c^*(a_i)$\;
% $P(\mathtt{T}) \gets$ Create an ASP program from input problem with makespan T\;% $M \gets \{\}$\;
%  \While{$M$ is empty}{%
%   $M \gets$ Solve $P(T)$\;
%   \If{$M$ is not empty}{
%     $C(M) \gets$ cost of the solution of model $M$\;
%     $\Delta \gets C(M) - 1 - C^-$\;
%     $M^* \gets$ Solve $P(\mathtt{T}+\Delta)$\;
%     \Return{M}\;
%   }
%   $\mathtt{T} \gets \mathtt{T}+1$\;
 %}

%\end{algorithm}

\section{Empirical Evaluation}

\section{Conclusions and Future Work}
We proposed a compilation to ASP that produces cost-optimal solutions, which is competitive with other state of the art algorithms on congested grids.

Future research will be focused on different aspects:
\begin{itemize}
    \item Because the grounding process takes a lot of time in the algorithm, we will research alternatives to execute the \emph{grounding}. This can mean doing this process manually.
        
    \item Research the possibility on making a sub optimal solver based on our encodings. Debido a los alentadores resultados mostrados en grillas pequeñas es que podríamos integrar nuestro solver a otros algoritmos como Windowed Hierarchical Cooperative A*. De esta manera podemos aprovechar los beneficios de los algoritmos basados en busqueda en grillas de gran tamaño  y los beneficios de nuestra codificación en problemas compactos.

\end{itemize}

\footnotesize
\bibliographystyle{aaai}
\bibliography{ref}

\end{document}
