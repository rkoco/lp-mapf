\section{A Linear Encoding}
The encoding we have proposed is quadratic in the number of agents. In this section we show how to make it linear by introducing new atoms to the encoding. Specifically, we introduce the following atoms:
\begin{itemize}
    \item $rt(x,y,t)$ (resp.\ $lt(x,y,t)$) which specifies that the edge between $(x,y)$ and $(x,y+1)$ was traversed by \emph{some} agent at time $t$ from left to right (resp.\ from right to left).
    \item $ut(x,y,t)$ (resp.\ $dt(x,y,t)$) which specifies that the edge between $(x,y)$ and $(x,y+1)$ was traversed upwards (resp.\ downwards) by some agent at time $t$.
    \item $st(x,y,t)$, indicates that some agent stayed at $(x,y)$, that is, it  performed a wait action at time $t$.
\end{itemize}

The dynamics of these atoms are defined using one rule with variables. The rule for $rt$ is:
\begin{equation}\small\label{encoding:rt}
  rt(X,Y,T) \leftarrow exec(A,right,T), at(A,X,Y,T),
\end{equation}
while the rule for $st$ is:
{\small\begin{equation*}
  st(X,Y,T) \leftarrow at(A,X,Y,T), exec(A,wait,T).
\end{equation*}}
We omit the rules for $dt$, $lt$, and $ut$ since they are analogous to \eqref{encoding:rt}.

Using these predicates we now express the fact that a single cell cannot be entered at the same time instant by two different agents. This requires six rules each of which corresponds to a pairs of actions in $\{\Right,\Left,\Up,\Down\}$. For example, the following rule expresses that $(X,Y)$ cannot be entered by an agent performing a \Down action at the same time that is entered by another agent performing  \Up:
{\small\begin{equation*}
  \leftarrow lt(X,Y,T), dt(X,Y,T)
\end{equation*}}
It is easy to verify that these rules do not mention pairs of different agents, unlike \eqref{encoding:vertexone} and \eqref{encoding:swapone}, and as such after grounding we end with $\Theta(|\mathcal{A}|\cdot|V|\cdot\mathtt{T})$ rules, and therefore the resulting encoding is linear in $|\mathcal{A}|$.
%\section{Bounding the effect positions}
%We change the basic effect rules to include a bound on the positions of the agents:
%\begin{align*}
%    \at(A,X,Y,T) \leftarrow &at(A,X',Y',t), exec(A,M,T), \cup B, \\
%    &cost\_to\_go(A,X,Y,C), T + C <= \mathtt{T}
%\end{align*}

%Where $cost\_to\_go(A,X,Y,C)$ specify that $C$ is the minimum distance from the position $(X,Y)$ to the goal of agent $A$. This way we can ignore the generation of rules to positions that will not reach the goal.
